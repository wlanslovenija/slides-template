\documentclass[]{beamer}
\mode<presentation>
{
% use attribute "dark" for dark theme
% use attribute "contrast" for maximum text contrast
\usetheme[]{wlansi}
\usefonttheme[onlymath]{serif}
\setbeamercovered{transparent}
}  

\usepackage{hyperref}
\usepackage[utf8]{inputenc}
\usepackage{mathptmx}
\usepackage{wlansi} % The wlan slovenija package defines Mitar-approved fonts and slide layouts (has to be included after mathptmx

\title{Presentation sample}

%\subtitle {Include Only If Presentation Has a Subtitle}

%\author{Luka Čehovin}

%\institute{wlan slovenija}

\date[] {1.0}

\begin{document}

% makes a title slide
\maketitle

\simpleslide{}{This slide contains only the main text.}{}

\simpleslide{The top slot}{This slide contains the top slot text and the main text.}{}

\simpleslide{}{This slide contains the main text and the bottom slot text.}{The bottom slot}

\simpleslide{The top slot}{}{The bottom slot}

\simpleslide{The top slot}{This slide contains text in all slots.}{The bottom slot}

\simpleslideimage[height]{clipart/node_single_online.png}{This slide contains an image and some text.}

\simpleslideimage[width]{clipart/node_network_online.png}{This slide contains an image and some text.}

\fullslideimage{clipart/node_network_online.png}

\begin{frame}

This is how you make a custom (non-Mitar-approved) slide.

\begin{itemize}
\item item 1
\item item 2
\item item 3
\end{itemize}

\end{frame}

\begin{frame}

This is how you make a custom (non-Mitar-approved) slide with enumeration.

\begin{enumerate}
\item item 1
\item item 2
\item item 3
\end{enumerate}

\end{frame}

\simpleslide{}{That is all.}{Pretty simle, huh?}

\end{document}
